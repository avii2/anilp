
\section{Introduction}
The rapid increase in computing power has positioned machine learning (ML) as the default tool for many data-rich problems. Modern ML pipelines often rely on collecting massive amounts of user data in centralized servers to achieve competitive accuracy. This paradigm raises two major concerns: (i) collecting and storing petabyte-scale datasets is expensive in terms of computation, communication, and operational overhead, and (ii) the resulting repositories frequently contain sensitive user information, which heightens the risk of data leakage and erodes users' willingness to share their data.

Intrusion detection systems (IDS) exemplify this tension. Signature-based IDS classify malicious activity by matching events to a catalog of known attack vectors. While effective against previously analyzed threats, they are ill-suited for detecting novel or zero-day attacks whose fingerprints are absent from the signature database. Anomaly-based IDS mitigate this limitation by modeling normal traffic patterns statistically or through ML, yet they inherit the privacy and data-collection challenges of centralized learning.

Federated learning (FL) offers an attractive alternative by keeping raw data on user devices and aggregating only local model updates \citep{bcflsurvey}. In horizontal FL, clients retain samples that share the same feature space, whereas vertical FL partitions the feature space across clients operating on aligned samples. By decoupling learning from centralized data collection, FL addresses the privacy concerns that plague ML-based IDS while still benefiting from distributed data diversity.

However, canonical FL deployments still rely on a centralized aggregator to orchestrate training, which reintroduces a single point of failure and places trust in an entity that may behave maliciously or become unavailable. To overcome this limitation, we combine FL with a lightweight blockchain layer that records model updates on a distributed ledger. This hybrid design removes the central coordinator, provides tamper-evident provenance for every update, and enhances resilience against emerging security threats.
